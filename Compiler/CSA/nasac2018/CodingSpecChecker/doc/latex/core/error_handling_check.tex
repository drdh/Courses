\subsection{错误处理检测}
\label{sec:core:error}
该检测主要对带有错误返回值的函数进行检查。
由于并非所有的函数都带有错误返回值,
所以首先应该筛选出带有错误返回值的函数。
例如,标准库中的文件操作函数(fopen,fclose等)就属于此类。

本工具对需要检查的函数\(f\)的错误返回值检查方法如下:

\begin{itemize}
  \item
    对于单独出现的函数调用\(f\),
在其出现位置后的第一条语句应当立即对函数返回值进行检查,且为条件判断,
而不应当再调用其他的函数,否则给出警告。
  \item
    如果函数调用 \(f\)直接出现在条件语句中,
则判断为一种直接检查,不给出警告。

\end{itemize}

由于不同函数的返回值未知,
并且不同函数(第三方函数/用户自定义函数)的返回值所表现出来的意义不同,
所以无法准确判断对当前函数调用是否需要进行返回值检查。
为此,本参赛队引入一个配置文件名为funcName.txt,用于存放带有错误返回值函数的函数名,包括标准库中常用的带有错误返回的函数。
一旦在运行程序中出现的函数名与当前文件中的记录相吻合时,
就断定该函数带有错误返回值,并立即对其进行检查。
本工具允许用户自行扩充包含错误返回值的第三方函数和用户自定义函数的名字列表。
 
