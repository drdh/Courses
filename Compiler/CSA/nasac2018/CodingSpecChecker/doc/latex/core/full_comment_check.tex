\subsection{函数头注释检测}
\label{sec:core:comment}

在 Clang 提供的框架下,调用

  \ \ \ \ \href{https://clang.llvm.org/doxygen/classclang_1_1ASTContext.html#ac10b2ebc25da948d370e74f7688fd134}{ASTContext::getRawCommentForDeclNoCache()} 
\\即可获取函数的头注释内容。

在获得函数的头注释后,本工具进行如下的检查:

\begin{itemize}
  \item
  检查注释风格(“/**/”、“//”之一)是否与之前出现的注释风格一致,如果不一致则汇报这一情况,并且不再对其他函数头注释进行检查以减少分析所耗时间。

  \item 
  如果函数存在指针类型的参数,则在注释中寻找该参数名(严格匹配)。
  如果找到则认为用户已经对该指针相关的内存约定做了说明,否则提示需要对该参数进行说明。

  \item 
  如果函数返回指针类型的变量,则在注释中寻找“返回”字样。
  如果找到,则认为用户已经对返回值做了说明,否则提示需要对返回值做说明。

  \item
  在注释中寻找后跟空白的冒号(正则表达式 \texttt{(:\textbar{}:)(\textbackslash{}\textbackslash{}s+\textbar{}\$)})。
  如果找到,则认为存在空有格式的注释,产生提示。

\end{itemize}

在仔细考虑竞赛题目要求后,本参赛队发现:检查在注释中是否说明性能约束、算法实现、可重入性是很难判断的,如处理不当,很容易导致大量误报。
为此,本参赛队没有实现相关的功能,但是用户可以根据实际需求,
在本工具设计和实现的 FullCommentChecker 类中添加相关的代码方便地扩充功能。
