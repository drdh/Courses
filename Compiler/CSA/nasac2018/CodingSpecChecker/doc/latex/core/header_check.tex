\subsection{头文件检查}
\label{sec:core:header}

\subsubsection{头文件自包含检查}

如果一个头文件是自包含的,
那么单独以该头文件作为本工具输入应正常通过预处理、语法检查和语义检查
生成合法的AST结构;
否则说明该头文件可能存在对未定义结构的引用,即不是自包含的。
本工具针对该项检查并不需要进行额外的定义,
而只需要基于Clang提供的基本编译流程即可判断。

\subsubsection{头文件循环包含检查}

头文件循环引用检查将头文件间的引用关系维护为有向图$G=\{V,E\}$,
其中顶点集合$V$中的元素表示头文件,
有向边集合$E$中的元素表示头文件间的包含关系,
通过检查最终图$G$是否有环判断头文件是否循环包含。

头文件循环包含检查的实现是基于Clang的预处理器回调对象(\href{https://clang.llvm.org/doxygen/classclang_1_1PPCallbacks.html}{PPCallback})来定制的,
主要关注回调事件列表中的
头文件包含事件(\href{https://clang.llvm.org/doxygen/classclang_1_1PPCallbacks.html#ad5509ca394c21faaead91ec8add75dd2}{InclusionDirective})和主文件结束事件(\href{https://clang.llvm.org/doxygen/classclang_1_1PPCallbacks.html#a63e170d069e99bc1c9c7ea0f3bed8bcc}{EndOfMainFile})。
本工具在头文件包含事件回调函数中记录
当前发起包含的文件绝对路径$u$和被包含的文件绝对路径$v$,
将$u$和$v$加入顶点集合$V$,将$(u,v)$加入有向边集合$E$。
本工具在主文件(作为工具输入的头文件)结束事件回调函数中检测图$G$
是否有环存在。
如果存在,则将构成环的头文件节点通过诊断引擎输出。
