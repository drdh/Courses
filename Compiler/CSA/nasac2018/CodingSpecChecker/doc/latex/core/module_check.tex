\subsection{函数参数合法性检查}
\label{sec:core:arg}

该项检查针对竞赛要求第二点进行实现,本参赛队对第二点的需求理解为:
考虑编译单元$U$ (一个C 程序文件为一个编译单元)中的函数$F$,
当$F$向编译单元外暴露时,即$F$未被 $static$ 修饰,
需要检查 $F$ 的输入参数的合法性;
否则,不需要检查 $F$ 的输入参数的合法性。
当 $F$ 调用函数 $G$ 时,如果 $G$ 与 $F$ 都在同一个编译单元 $U$ 中且不对外暴露,
那么 $G$ 参数的合法性将由 $F$ 负责检查,
否则 $F$ 不需要保证 $G$ 调用参数的合法性检查。

在上述理解中,“检查”针对的对象是内存区域存储的右值。
若内存区域的右值直接作为分支条件表达式的操作数,
那么就认为相应内存区域的右值得到检查。

本工具基于{\bf 不信任传播}的思想实现上述需求。
首先,根据需求设置某些内存区域右值为不信任源,
如需要检查的函数形参右值;
然后,
由不信任的右值经运算后导出的其他右值也不被信任。
当不信任的右值被读取时,本工具将产生警告;
当程序代码有对不信任的内存区域右值的“检查”,
则该右值获得信任,并且该右值定值时所依赖的所有内存区域右值也获得信任。


本工具采用Clang提供的静态分析框架对输入的源C 程序文件进行
基于符号执行的路径敏感的过程内分析。
在分析中,追踪的程序点状态定义如\autoref{equ:module_check_state}所示。
其中$S_1$表示内存区域右值(Memory Region Values, 简记RV)的信任状态,
$T$表示信任,$U$表示不信任;
$S_2$表示内存区域右值到最近一次定值时依赖的内存区域右值集合。
% 前者主要是为了在读取内存区域右值是判断是否右值检查过,
% 后者的目的主要是为了支持间接检查的情况,
% 典型的情况如需要错误处理的函数调用,
% 当函数调用返回的错误指示尚未检查前,该函数岀参可能存在错误,被认为处于未检查状态;
% 当错误指示检查后,认为函数岀参相应的处于已检查状态。

\begin{equation}
\label{equ:module_check_state}
S = \{
S_1 : RVs \mapsto \{T, U\}, 
S_2 : RVs \mapsto \{RVs\}
\}
\end{equation}

在上述程序状态基础上,
定义如\autoref{fig:module_check} 所示的转换规则和检查动作,
其中函数$Value()$接收一个左值并返回对应右值,
函数 $DependentRegionValues()$ 接收一个表达式 $E$
并返回 $E$ 中直接依赖的内存区域右值。
\begin{figure}[p]
\begin{tabular}{l}
$[[Entry]]: $\\
$\qquad S_1 = \phi, S_2=\phi; $\\
$[[ParmDecl\ X]]: $\\
$\qquad S_1[Value(X) \mapsto IsStatic(Func)\ ?\ T : U]; $\\
$[[VarDecl\ X]]: $\\
$\qquad S_1[Value(X) \mapsto U]; $\\
$[[Condition\ E]]: $\\
$\qquad \forall RV \in DependentRegionValues(E), S_1[RV \mapsto T]; $\\
$\qquad \forall RV' \in S_2(RV), S_1[RV' \mapsto T]; $\\
$Pre.[[E(E_1,E_2,...,E_n)]]: $\\
$\qquad \text{if}\ IsNotExtern(E),$\\
$\qquad \text{then} \forall E_i \in \{E_1,E_2,...,E_n\} \land S_1(E_i) = U,
\text{report unchecked argument}\ E_i; $\\
$Post.[[E(E_1,E_2,...,E_n)]]: $\\
$\qquad \text{if}\ NeedHandleError(E), $\\
$\qquad \text{then}\ \forall E_i \in \{E_1,E_2,...,E_n\} \land IsOutput(E, E_i), S_1[Value(E_i) \mapsto U]; $\\
$\qquad S_2[Value(E(E_1,E_2,...,E_n)) \mapsto DependentRegionValues(E(E_1,E_2,...,E_n))] $\\
$[[E_1 = E_2]]: $\\
$\qquad S_1[Value(E_1) \mapsto \forall RV \in DependentRegionValues(E), S_1(RV) = T\ ?\ T : U]; $\\
$\qquad S_2[Value(E_1) \mapsto DependentRegionValues(E_2)] $\\
$[[Read\ E]]: $\\
$\qquad \text{if}\ S_1(Value(E)) = U,\text{then report read before check};$
\end{tabular}
\caption{函数参数合法性检查状态转移与检查动作} \label{fig:module_check} 
\end{figure}

