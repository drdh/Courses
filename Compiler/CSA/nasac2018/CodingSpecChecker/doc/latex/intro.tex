\section{工具简介}
\label{sec:intro}
``违反编码规范的缺陷检测工具''(以下简称本工具)是根据NASAC2018软件原型命题竞赛组 
2018年10月21日晚发布的竞赛题目,
由中国科学技术大学计算机科学与技术学院张昱老师于10月22日
联络2018级硕士研究生张宇翔组队研制。
本参赛队由队长张宇翔、2016级本科生邓胜亮(10月23日加入)、
2018级硕士研究生李森(10月26日加入)组成,
工具研发的起止时间为2018年10月22日至11月10日;
张昱老师进行全程指导。

本参赛队利用 \href{https://clang.llvm.org/}{Clang 7.0.0} 
提供的工具框架
设计实现了具有可配置、易扩展的自动化违反编码规范的缺陷检测工具。
该工具提供函数参数合法性检查、函数错误返回码处理检查、
变量按需初始化检查、头文件自包含检查、循环依赖检查、
函数头注释检查、标识符命名检查等,
产生的诊断信息内容全面、可读性强。
用户可以方便地配置工具的行为,扩展其功能。

\subsection{功能简介}
\label{sec:intro:func}
该工具能够进行的检查如下:

\begin{itemize}
\item {\bf 函数参数合法性检查}
检查代码中模块对外接口是否进行了合法性检查。

\item{\bf 函数错误返回码处理检查}
检查在调用提供了错误指示机制的函数后,是否立即检查了错误指示。

\item{\bf 变量按需初始化检查}
检查代码中是否存在冗余的初始化。

\item{\bf 头文件自包含及循环包含检查}
检查头文件是否自包含且没有循环依赖。

\item{\bf 函数头注释检查}
检查函数头注释风格是否一致、注释中是否只有格式没有内容、是否说明了必要的信息、未注释的函数是否需要补充注释。

\item{\bf 标识符命名检查}
指出函数、变量名中的非英文单词,并对英文变量名提供常见的缩写建议。
\end{itemize}

所有检查的结果在汇报时都会指出其在源码中对应的位置(包括文件路径、行号、列号),方便用户进行修改。
各功能具体的实现细节将在 \S\ref{sec:core} 进行介绍。

\subsection{使用说明简介}
\label{sec:intro:usage}

该工具为命令行工具,用户通过提供命令行参数,对工具进行以下配置:

\begin{itemize}
    \item 关闭指定的一个或多个检查;
    \item 指定英文词典、缩写词典的位置;
    \item 指定返回错误码的函数列表的位置;
    \item 指定要检查的项目的编译命令列表位置;
    \item 指定要检查的文件或目录;
\end{itemize}

除了提供命令行选项以外,用户可以自行删减、扩充英文词典、缩写词典、返回错误码的函数列表,
从而方便地根据实际需要来定制工具的行为。

关于以上配置方式的具体说明见 \S\ref{sec:usage} 一节。
