\subsubsection{2018.11.09组会}
\label{sec:g2018:1026}
\bhei{出席人员}:\RenZH、\TangLT、\ZY、\ZhangYX

\bhei{记录}:\TangLT、\ZY

\sect{工作情况}
\begin{itemize}
\item \RenZH:
    \begin{itemize}
        \item 概率图模型大致有哪些
        \item PyMC3和Edward中概率编程、模型建立的区别和具体代码
        \item 提出了概率编程需要的一些支持
            \begin{itemize}
        		\item 模型中的属性关系构建和表示
       		 \item 内嵌的采样、推断算法
   	    \end{itemize}
        \item 提出的一个问题
            \begin{itemize}
                \item 概率编程在机器学习模型构建中与其他模块之间的关系?
            \end{itemize}
    \end{itemize}
\item \TangLT:
    \begin{itemize}
        \item 进一步了解PyMC3
        \begin{itemize}
        	\item 变量类型
       		 \item 模型建立过程
		 \item 如何进行推断:利用后验分布的采样
   	    \end{itemize}
        \item 采样方法简介
        \begin{itemize}
        	\item 为什么需要采样
       		 \item 常见分布采样
		 \item 拒接采样
		 \item 重要性采样
		 \item MCMC的具体实现和缺陷
   	    \end{itemize}
    \end{itemize}
\end{itemize}

\sect{下步工作建议}
\begin{itemize}
\item \RenZH:
    \begin{itemize}
    	\item 了解PRISM的语法和构建过程
     \end{itemize}
\item \TangLT:
    \begin{itemize}
        \item 具体了解各种概率图模型及其算法
    \end{itemize}
\end{itemize}