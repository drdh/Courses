%段落缩进
\usepackage{indentfirst}
\setlength{\parindent}{20pt}
%中文设置
% \usepackage[cm-default]{fontspec}
% \usepackage{xeCJK}
% \defaultfontfeatures{Mapping=tex-text}
% 中文断行
% \XeTeXlinebreaklocale "zh"
% \XeTeXlinebreakskip = 0pt plus 1pt minus 0.1pt
% 行距
% \linespread{1.25} 
%将系统字体名映射为逻辑字体名称
% \newcommand\fontnamesong{Songti SC}
% \newcommand\fontnamehei{PingFang SC}
% \newcommand\fontnamekai{Kaiti SC}
% \setCJKmainfont{Kaiti SC}
%\setsansfont{LMSans10}
%楷体
% \newcommand{\kai}[1]{{\KAI#1}}
%黑体
% \setCJKfamilyfont{HEI}{PingFang SC}
% \newcommand\hei[1]{{\CJKfamily{HEI}#1}}
% \newcommand\bhei[1]{{\CJKfontspec{PingFang SC Heavy}{#1}}}
%英文
\newcommand{\en}[1]{\,{\ENF#1}\,}
\newcommand{\ensf}[1]{\,{\ENSF#1}\,}
%减少itemize的行间距
\usepackage{paralist} 
\let\itemize\compactitem 
\let\enditemize\endcompactitem 
\let\enumerate\compactenum 
\let\endenumerate\endcompactenum 
\let\description\compactdesc 
\let\enddescription\endcompactdesc
%作者之间的连接
\renewcommand\Authand{ \quad }
%目录
\renewcommand\contentsname{\textbf{目 录}} 
\if\hasalgo
\floatname{algorithm}{\textbf{算 法}} 
\fi
\renewcommand\figurename{\textbf{图}} 
\renewcommand\figureautorefname{图} 

\renewcommand\tablename{\textbf{表}} 
\renewcommand{\refname}{\textbf{参 考}}
\renewcommand{\lstlistingname}{\textbf{代码}} %% 重命名Listings标题头
%

\newcommand\sect[1]{\paragraph{\textbf{#1}}~~}
\newcommand\subsect[1]{\vspace{.2cm}{\noindent\textbf{#1}}\vspace{.2cm}}

