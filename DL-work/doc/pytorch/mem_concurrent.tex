\subsection{pytorch 内存管理和并发}

\subsubsection{cpu allocator}

位于\href{https://github.com/pytorch/pytorch/blob/master/c10/core/CPUAllocator.cpp}{c10/core/CPUAllocator.cpp}

\begin{itemize}
\tightlist
\item
  malloc:位于:\href{https://github.com/pytorch/pytorch/blob/master/c10/core/CPUAllocator.cpp\#L40}{alloc\_cpu}

  \begin{itemize}
  \tightlist
  \item
    如果是Linux:使用系统函数posix\_memalign,以gAlignment=64 byte对齐
  \item
    同时如果是\href{http://www.cnblogs.com/shanyou/archive/2009/12/26/1633052.html}{NUMA架构},那么还会将分配的内存移到对应的节点
  \item
    \href{http://www.cnblogs.com/shanyou/archive/2009/12/26/1633052.html}{NUMA架构}:一种非一致存储器访问架构,访问本节点(CPU)内的存储器所需要的时间,比访问某些远程节点内的存储器所花的时间要少得多。
  \end{itemize}
\item
  free:位于:\href{https://github.com/pytorch/pytorch/blob/master/c10/core/CPUAllocator.cpp\#L79}{free\_cpu}

  \begin{itemize}
  \tightlist
  \item
    直接free
  \end{itemize}
\end{itemize}

\subsubsection{cuda caching allocator}

位于\href{https://github.com/pytorch/pytorch/blob/master/c10/cuda/CUDACachingAllocator.cpp}{c10/cuda/CUDACachingAllocator.cpp}

\begin{itemize}
\item
  cuda的内存管理器
\item
  内存分配是跟stream相关的,一个stream中被释放的内存只会再次分配给这个stream

  \begin{itemize}
  \tightlist
  \item
    stream是对cuda的cuStream的包装,对cuStream做了一些额外处理
  \item
    位于:\href{https://github.com/pytorch/pytorch/blob/master/c10/cuda/CUDAStream.h}{c10/cuda/CUDAStream.h}
  \end{itemize}
\item
  分成两种block,small block,large block

  \begin{itemize}
  \tightlist
  \item
    small
    block:最大:kSmallSize=1MB,一开始使用\texttt{cudaMalloc}分配时为kSmallBuffer
    = 2MB
  \item
    large block:一开始使用\texttt{cudaMalloc}最少被分配kLargeBuffer=
    20MB,或者是kRoundLarge = 2MB的倍数,取决于需要的内存大小
  \item
    一开始分配内存大小,具体见:\href{https://github.com/pytorch/pytorch/blob/master/c10/cuda/CUDACachingAllocator.cpp\#L454}{get\_allocation\_size}
  \end{itemize}
\item
  large block和small block是在不同的存储池(pool)里
\item
  这些block的创建只在第1次请求block时创建,也即为lazily created
\item
  malloc:位于:\href{https://github.com/pytorch/pytorch/blob/master/c10/cuda/CUDACachingAllocator.cpp\#L192}{malloc}

  \begin{itemize}
  \item
    分配器先试图找到满足需求的small block,如果没有,再找large
    block,如果block比需要的大,那么它可能会被分成两块
  \item
    如果找不到,使用FreeCudaMemoryCallbacksRegistry

    \begin{quote}
    Caching allocator will execute every registered callback if it
    unable to find

    block inside of already allocated area.
    \end{quote}
  \item
    不行的话使用
    \texttt{cudaMalloc},如果失败,free所有cache的block(只free当前需要分配内存的设备内存),再试一次,失败,报错
  \item
    分成两块

    \begin{itemize}
    \tightlist
    \item
      small block分块:如果剩下的大于最小分配单元(512byte)
    \item
      large block分块:如果剩下的大于最大的small block(1MB)
    \item
      分成两块时:剩下的块在被分配的块后面,以链表组织
    \end{itemize}
  \end{itemize}
\item
  free:位于:\href{https://github.com/pytorch/pytorch/blob/master/c10/cuda/CUDACachingAllocator.cpp\#L303}{free}

  \begin{itemize}
  \tightlist
  \item
    分配器会试图将被释放的块和其前驱或后继进行合并,如果存在前驱或后继且没有被分配的话
  \end{itemize}
\item
  所有block以512 byte (kMinBlockSize)为单元进行分配
\item
  一些参数
\end{itemize}

\begin{lstlisting}
constexpr size_t kMinBlockSize = 512;       // all sizes are rounded to at least 512 bytes
constexpr size_t kSmallSize = 1048576;      // largest "small" allocation is 1 MiB
constexpr size_t kSmallBuffer = 2097152;    // "small" allocations are packed in 2 MiB blocks
constexpr size_t kLargeBuffer = 20971520;   // "large" allocations may be packed in 20 MiB blocks
constexpr size_t kMinLargeAlloc = 10485760; // allocations between 1 and 10 MiB may use kLargeBuffer
constexpr size_t kRoundLarge = 2097152;     // round up large allocs to 2 MiB
\end{lstlisting}

\subsubsection{pinned(Page-Locked) memory allocator}

位于\href{https://github.com/pytorch/pytorch/blob/master/aten/src/THC/THCCachingHostAllocator.h}{aten/src/THC/THCCachingHostAllocator.h}和\href{https://github.com/pytorch/pytorch/blob/master/aten/src/THC/THCCachingHostAllocator.cpp}{aten/src/THC/THCCachingHostAllocator.cpp}

\begin{itemize}
\item
  pin memory 使用DMA
  engines进行数据传输,在cuda上面对不同stream进行数据传输时可以overlap(重叠,即同时进行),而且os不会对这块内存进行分页
\item
  官方有一个note:\href{https://pytorch.org/docs/stable/notes/cuda.html\#use-pinned-memory-buffers}{Use
  pinned memory buffers}
\item
  这个allocator如果频繁使用可能会造成严重的性能问题,pinning是一个expensive
  operation

  \begin{itemize}
  \item
    在\href{https://www.cs.virginia.edu/~mwb7w/cuda_support/pinned_tradeoff.html}{这篇文章}中提到:

    \begin{quote}
    pinned memory is much more expensive to allocate and deallocate but
    provides higher transfer throughput for large memory transfers.
    \end{quote}
  \end{itemize}
\item
  和caching allocator相同之处:

  \begin{itemize}
  \tightlist
  \item
    都会把之前alloc然后被free的内存cache下来,而不会立即free掉
  \item
    每次请求内存时先从cache的内存里面找有没有满足的,没有再使用cudaHostAlloc分配内存
  \end{itemize}
\item
  和caching allocator不同之处在于:

  \begin{itemize}
  \tightlist
  \item
    使用cudaHostAlloc分配内存而不是cudaMalloc
  \item
    没有对请求的size进行对齐操作,而且也不会把大的block(cache的)分成小的block
  \end{itemize}
\end{itemize}

\hypertarget{cuda-stream}{%
\subsubsection{CUDA Stream}\label{cuda-stream}}

\begin{itemize}
\item
  stream目前的理解是对应GPU设备的一个任务队列,在相同流的任务是隐式同步的(Implicit
  Synchronization),在不同的流之中任务可以异步执行(并发,concurrent)
\item
  异步或者说并发性取决于硬件的支持(下面的数据传输如果要支持这种特性,host
  memory必须是page-locked,也即上面的pin memory)

  \begin{itemize}
  \tightlist
  \item
    是否支持并发的kernel执行
  \item
    是否支持并发的数据传输
  \item
    是否支持kernel执行和数据传输重叠(overlap)
  \end{itemize}
\item
  stream还有优先级(priority)之分

  \begin{itemize}
  \item
    按照\href{https://docs.nvidia.com/cuda/cuda-runtime-api/group__CUDART__STREAM.html\#group__CUDART__STREAM_1ge2be9e9858849bf62ba4a8b66d1c3540}{官方cudaStreamCreateWithPriority
    API的说法}

    \begin{quote}
    Work in a higher priority stream may preempt work already executing
    in a low priority stream.
    \end{quote}

    \begin{quote}
    In the current implementation, only compute kernels launched in
    priority streams are affected by the stream's priority. Stream
    priorities have no effect on host-to-device and device-to-host
    memory operations.
    \end{quote}
  \item
    在调度时,优先级高的可能会preempt(占据,先于)正在执行的低优先级任务
  \item
    对于内存操作没有影响
  \end{itemize}
\item
  一个thread如果没有创建stream,那么会有一个default stream

  \begin{itemize}
  \tightlist
  \item
    这个default stream是隐式同步的(Implicit Synchronization)
  \item
    这个stream不会和其他stream overlap(同时执行)
  \end{itemize}
\end{itemize}

\hypertarget{pytorch-cuda-stream}{%
\subsubsection{pytorch cuda stream}\label{pytorch-cuda-stream}}

位于\href{https://github.com/pytorch/pytorch/blob/master/c10/cuda/CUDAStream.h}{c10/cuda/CUDAStream.h}和\href{https://github.com/pytorch/pytorch/blob/master/c10/cuda/CUDAStream.cpp}{c10/cuda/CUDAStream.cpp}

\begin{itemize}
\tightlist
\item
  每个device有3个stream pool(为了降低stream创建和销毁的开销)

  \begin{itemize}
  \tightlist
  \item
    default pool:只包含default stream,由于default
    stream是默认存在的,所以并不用创建(nullptr 的stream即为default
    stream)
  \item
    low priority pool:包含32个low
    priority的stream,创建时使用的优先级为low,在请求一个stream时返回的是一个循环的index,即33次请求过程中,第1次index=0,第2次index=1,第32次index=31,第33次index=0
  \item
    high priority pool:同low priority
    pool,只不过创建时使用的优先级为high
  \item
    这些stream的创建只在第1次请求stream时创建,也即为lazily created
  \end{itemize}
\item
  HIP(ROCm架构,AMD
  GPU),不支持priority的stream,所以high和low的pool是一样的
\item
  不同的thread对于这些stream是共享的(除了default
  stream),不同thread可以enqueue
  kernel到同一个stream,这使得可以同步这些kernel,当然,由于kernel的enqueue顺序不定,所以如果想达到这个目标需要在host端进行同步的enqueue操作
\end{itemize}

\subsubsection{参考文章}

\begin{itemize}
\tightlist
\item
  CUDA官方指南:\href{https://docs.nvidia.com/cuda/cuda-c-programming-guide/index.html\#abstract}{CUDA
  C Programming Guide}
\item
  一份比较好的介绍cuda流及其并发的ppt(应该也是CUDA官方的):
  \href{http://on-demand.gputechconf.com/gtc/2014/presentations/S4158-cuda-streams-best-practices-common-pitfalls.pdf}{cuda-streams-best-practices-common-pitfalls}
\item
  HIP,ROCm架构:https://rocm.github.io/和https://github.com/ROCm-Developer-Tools/HIP
\end{itemize}

\subsubsection{在看源码时遇到的一些问题}

\begin{itemize}
\item
  源码里面使用了很多macro,registry,dispatcher,hook之类的一些东西(我觉得应该是一些编程模型),还有使用了很多c++
  11的新特性,导致我看不太懂,打算了解一下
\item
  dispatcher:\href{https://github.com/pytorch/pytorch/blob/master/aten/src/ATen/core/dispatch/README.md}{aten/src/ATen/core/dispatch/README.md}
\item
  registry:\href{https://github.com/pytorch/pytorch/blob/master/c10/util/Registry.h}{c10/util/Registry.h}
\item
  hook:\href{https://github.com/pytorch/pytorch/blob/master/torch/csrc/autograd/function_hook.h}{torch/csrc/autograd/function\_hook.h},\href{https://github.com/pytorch/pytorch/blob/master/torch/csrc/autograd/python_hook.cpp}{torch/csrc/autograd/python\_hook.cpp}
\end{itemize}
