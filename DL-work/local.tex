
%%%%%%%%%%%%%%%%%%%%%%%%%%%%%%%%%%%%%%%%%%%%%%%%%%%%%%%%%%%%%%%%%%%%%%%%%%%%%
%  Including standard latex packages
%%%%%%%%%%%%%%%%%%%%%%%%%%%%%%%%%%%%%%%%%%%%%%%%%%%%%%%%%%%%%%%%%%%%%%%%%%%%%
%

\usepackage{stmaryrd}            %more math symbols
\usepackage{amsmath}             %ams math
\usepackage{amsfonts}            %ams fonts
\usepackage{amssymb}             %amsmath
\usepackage{mathrsfs}            % \mathscr
\usepackage{latexsym}            %latex symbols
\usepackage{graphicx}            %grahpicx
\usepackage{tabularx}
\usepackage{subfigure}
\usepackage{wrapfig}
\usepackage{multirow}
\usepackage{multicol}
\usepackage{cite}                % sort citation numbers
\usepackage{url}
\usepackage{authblk}
\usepackage{alltt}
\usepackage{import}
\usepackage{rotating}
\usepackage{fancyvrb}

\usepackage{geometry}
\geometry{left=2.0cm, right=2.0cm, top=2.5cm, bottom=2.5cm}
\usepackage{longtable} % table
\usepackage{booktabs} % table
\usepackage[bookmarks,colorlinks]{hyperref}

%algorithm
\if\hasalgo
\usepackage{algorithmwh}
\fi

\usepackage[all]{xy}             %various figures
%\usepackage[amsmath,thmmarks]{ntheorem}%theorems
\usepackage{float}
%\usepackage[usenames,dvipsnames]{color}
\usepackage{xcolor}
\usepackage{xspace}
\usepackage[final]{listings}
%\usepackage[pdftex]{hyperref}
\usepackage{hyperref}
\hypersetup{
	bookmarksnumbered,%
	bookmarksopen,%
	colorlinks,%
	citecolor=blue,%
	filecolor=magenta,%
	linkcolor=blue,%
	%urlcolor=green,%
	hyperindex,%
	plainpages=false,%
	pdfstartview=FitH
}
\providecommand{\tightlist}{%
  \setlength{\itemsep}{0pt}\setlength{\parskip}{0pt}}

\if\showchange
\newcommand\greent[1]{\textcolor{green}{#1}}
\newcommand\bluet[1]{\textcolor{blue}{#1}}
\newcommand\redt[1]{\textcolor{red}{#1}}
\else
\newcommand\bluet[1]{#1}
\newcommand\redt[1]{#1}
\newcommand\greent[1]{#1}
\fi

\newcommand{\K}[1]{{\bf \textcolor{red}{#1}}}        % keyword
\newcommand\Q{{\textcolor{red}{Q.~}}}
\newcommand\A{{\textcolor{red}{A.~}}}
\newcommand\note[3]{{\textcolor{red}{[#1,#2}}:#3\textcolor{red}{]}}
\newcommand\codet[1]{\textcolor{brown}{\texttt{#1}}}
\newlength\savewidth
\newcommand\whline{\noalign{\global\savewidth\arrayrulewidth
                            \global\arrayrulewidth 1pt}%
                   \hline
                   \noalign{\global\arrayrulewidth\savewidth}}
%%%%%%%%%%%%%%%%%%%%%%%%%%%%%%%%%%%%%%%%%%%%%%%%%%%%%%%%%%%%%%%%%%%%%%%%%%%%%
%  Including all the local shared latex packages and macros
%%%%%%%%%%%%%%%%%%%%%%%%%%%%%%%%%%%%%%%%%%%%%%%%%%%%%%%%%%%%%%%%%%%%%%%%%%%%%
\renewcommand{\ttdefault}{cmtt}
\let\cmttdfl\ttdefault          %what for?
\let\cmsfdfl\sfdefault          %what for?

\newcommand{\kw}[1]{{\em #1}}        % keyword
\newcommand\etal{\emph{et al.\ }}
\newcommand\eg{\emph{e.g.,\ }}
\newcommand\ie{\emph{i.e.,\ }}
\newcommand\etc{\emph{etc.\ }}
%%%%%%%%%%%%%%%%%%%%%%%%%%%%%%%%%%%%%%%%%%%%%%%%%%%%%%%%%%%%%%%%%%%%%%%%%%%%%
% THE START OF THE MAIN DOCUMENT
%%%%%%%%%%%%%%%%%%%%%%%%%%%%%%%%%%%%%%%%%%%%%%%%%%%%%%%%%%%%%%%%%%%%%%%%%%%%%
\lstset{language=C,%
	basicstyle=\footnotesize\ttfamily,%\small,%
	columns=flexible,%
	keepspaces=true,%
	keywordstyle=\color{blue!70}, 
	commentstyle=\color{red!50!green!50!blue!90},
	rulesepcolor=\color{red!20!green!20!blue!20},
	lineskip=2pt,%
	%boxpos=c,%
	tabsize=2,%
	frame=single,%
	numbers=left,%
	numberstyle=\tiny,%
	xleftmargin=2em,xrightmargin=1em, aboveskip=1em,
	escapeinside=``,
	showspaces=false,
	breaklines=true,%这条命令可以让LaTeX自动将长的代码行换行排版
	breakautoindent=true,% 
	breakindent=4em, %
	extendedchars=false %这一条命令可以解决代码跨页时,章节标题,页眉等汉字>不显示的问题
}

% Choose abbreviated or long-version alternatives in paper
\long\def\abbr#1#2{#1}                  % abbreviated version
%\long\def\abbr#1#2{#2}                 % long version

% Choose abbreviations or long names/titles in bibliography
%\def\bibbrev#1#2{#1}                    % short version
%\def\bibbrev#1#2{#2}                   % long version
\def\bibbrev#1#2{Proceedings of the #2(#1)}                   % long version
%\def\bibbrev#1#2{\abbr{#1}{#2}}                % follow abbr macro

% Abbreviated or full citation lists: \abcite{basic}{others}
\newcommand{\abcite}[2]{\abbr{\cite{#1}}{\cite{#1,#2}}}

% Conference abbreviations: \bibconf[Nth]{SOSP}{Symposium on ...}
\newcommand{\bibconf}[3][]{Proceedings of the #1 #3 (#2)}

