\chapter{数据结构设计}
\section{逻辑结构设计}
\subsection{客户端数据结构}
客户端在运行过程中,需要维护的信息如下:

生命周期为整个程序运行过程的信息有:

\begin{itemize}
    \item 用户名,如果尚未登录则为空
    \item 当前请求
    \item 请求结果
\end{itemize}

生命周期为某个请求处理过程的信息有:

\begin{enumerate}
    \item 普通用户应用管理:数据结构1\\
        \begin{itemize}
            \item 目标应用名
        \end{itemize}
    \item 普通用户信息反馈:数据结构2\\
        \begin{itemize}
            \item 目标应用名
            \item 评价或评分
        \end{itemize}
    \item 开发者应用管理:数据结构3\\
        \begin{itemize}
            \item 目标应用名
        \end{itemize}
    \item 开发者信息反馈:数据结构4\\
        \begin{itemize}
            \item 目标应用名
            \item 回复的评论
            \item 银行账号
        \end{itemize}
\end{enumerate}

根据具体的请求,以上的数据结构还可再进一步细化。

\subsection{服务器端数据结构}
数据部分使用Oracle数据库存储,并不需要特别的数据结构。

\section{物理结构设计}
各数据结构无特殊物理结构要求。

\section{数据结构与程序模块的关系}
[此处指的是不同的数据结构分配到哪些模块去实现。可按不同的端拆分此表]
\begin{table}[htbp]
\centering
\caption{数据结构与程序代码的关系表} \label{tab:datastructure-module}
\begin{tabular}{|c|c|c|c|c|}
    \hline
    · & 普通用户应用管理 & 普通用户信息反馈 & 开发者应用管理 & 开发者信息反馈\\
    \hline
    数据结构1 & Y & · & · & ·\\
    \hline
    数据结构2 & · & Y & · & ·\\
    \hline
    数据结构3 & · & · & Y & ·\\
    \hline
    数据结构4 & · & · & · & Y\\
    \hline
\end{tabular}
\note{各项数据结构的实现与各个程序模块的分配关系}
\end{table}