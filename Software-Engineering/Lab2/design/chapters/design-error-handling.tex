\chapter{出错处理设计}

\section{客户端出错处理}
客户端的错误可以分为两类:由于网络异常导致的和服务器通信中断、用户发出的请求未能得到满足。

第一类错误应提醒用户检查网络状态,第二类错误是因为用户不具备完成其请求的条件,如试图评价一个用户不拥有的应用,
此类错误由服务器返回,提醒用户重新操作即可。

另外,客户端可能会由于未知的原因而崩溃,对于这类问题可以设置一个日志来存储客户端的行为,以便重启客户端后进行恢复以及
向服务器发送崩溃记录,但也可不做处理,因为应用商店不是类似办公软件的程序,不会涉及到具有很大价值的数据。

\section{服务器端出错处理}
服务器有两种避免重大错误的方式,一是使用log, 二是数据库采用分布式冗余设计。

紧急错误出现时,立即停止系统,并且上线备份系统。

通常的错误可以使用log自动或者人工修正,管理员可以添加自动修正规则。

非常重大的人为错误或者物理失效,仍然有分布式冗余的数据库作为保证。