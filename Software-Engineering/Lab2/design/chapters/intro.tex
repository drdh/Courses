\chapter{引言}
\section{编写目的}
在本项目的前一阶段,也就是需求分析阶段,已经将系统用户对本系统的需求做了详细的阐述,这些用户需求已经在上一阶段中对不同用户所提出的不同功能,实现的各种效果做了调研工作,并在需求规格说明书中得到详尽得叙述及阐明。

本阶段已在系统的需求分析的基础上,对应用商店系统做概要设计。主要解决了实现应用商店系统的程序模块设计问题。包括如何把该系统划分成若干个模块、决定各个模块之间的接口、模块之间传递的信息,以及数据结构、模块结构的设计等。在以下的概要设计报告中将对在本阶段中对系统所做的所有概要设计进行详细的说明,在设计过程中起到了提纲挈领的作用。

在下一阶段的详细设计中,程序设计员可参考此概要设计报告,在概要设计应用商店系统所做的模块结构设计的基础上,对系统进行详细设计。在以后的软件测试以及软件维护阶段也可参考此说明书,以便于了解在概要设计过程中所完成的各模块设计结构,或在修改时找出在本阶段设计的不足或错误。


\section{项目背景}
应用商店的开发与时俱进,现在出现的 PC 端与手机端应用商店出现了严重
的割裂,而且应用的质量得不到保证、用户的反馈得不到及时的回应、开发者的合理权益得不到保障、用户需要专门花时间去找应用等等问题层出不穷。

现在出现的应用商店无法解决这样问题,这也是本项目出现的主要原因,即
构建一个面向未来的、可拓展的、友好的应用商店系统。以达到全面替代当前的各种应用商店的目的。

\section{术语}
\begin{table}[htbp]
\centering
\caption{术语表} \label{tab:terminology}
\begin{tabular}{|c|c|}
    \hline
    缩写、术语 & 解释 \\
    \hline
    用户 & 应用使用人员 \\
    \hline
    开发者 & 应用开发人员 \\
    \hline
    管理员 & 服务器与数据库维护人员 \\
    \hline
\end{tabular}
% \note{这里是表的注释}
\end{table}