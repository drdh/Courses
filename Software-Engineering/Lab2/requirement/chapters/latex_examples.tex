\chapter{Latex使用例子}

\section{图}
\subsection{示例}
\begin{figure}[ht]
\centering
\includegraphics[width=10cm]{ustc_logo_fig}
\caption{测试图片} \label{fig:figure1}
\end{figure}

\subsection{带图注的图}
\begin{figure}[ht]
\centering
\includegraphics[width=10cm]{ustc_logo_fig}
\caption{带图注的图片}\label{fig:noted-figure}
\note{the solid lines represent the time histogram of the spontaneous activities of an old monkey cell(gray) and a young monkey cell (black). The bin-width is 1}
\end{figure}

\section{表格}

\subsection{A Simple Table}
\begin{table}[htbp]
\centering
\caption{这里是表的标题} \label{tab:simpletable}
\begin{tabular}{|c|c|}
    \hline
    a & b \\
    \hline
    c & d \\
    \hline
\end{tabular}
\note{这里是表的注释}
\end{table}

\subsection{长表格}
\begin{longtable}{ccc}
% 首页表头
\caption[长表格演示]{长表格演示} \label{tab:longtable} \\
\toprule[1.5pt]
名称  & 说明 & 备注\\
\midrule[1pt]
\endfirsthead
% 续页表头
\caption[]{长表格演示(续)} \\
\toprule[1.5pt]
名称  & 说明 & 备注 \\
\midrule[1pt]
\endhead
% 首页表尾
\hline
\multicolumn{3}{r}{\small 续下页}
\endfoot
% 续页表尾
\bottomrule[1.5pt]
\endlastfoot

AAAAAAAAAAAA   &   BBBBBBBBBBB   &   CCCCCCCCCCCCCC   \\
AAAAAAAAAAAA   &   BBBBBBBBBBB   &   CCCCCCCCCCCCCC   \\
AAAAAAAAAAAA   &   BBBBBBBBBBB   &   CCCCCCCCCCCCCC   \\
AAAAAAAAAAAA   &   BBBBBBBBBBB   &   CCCCCCCCCCCCCC   \\
AAAAAAAAAAAA   &   BBBBBBBBBBB   &   CCCCCCCCCCCCCC   \\
AAAAAAAAAAAA   &   BBBBBBBBBBB   &   CCCCCCCCCCCCCC   \\
AAAAAAAAAAAA   &   BBBBBBBBBBB   &   CCCCCCCCCCCCCC   \\
AAAAAAAAAAAA   &   BBBBBBBBBBB   &   CCCCCCCCCCCCCC   \\
AAAAAAAAAAAA   &   BBBBBBBBBBB   &   CCCCCCCCCCCCCC   \\
AAAAAAAAAAAA   &   BBBBBBBBBBB   &   CCCCCCCCCCCCCC   \\
AAAAAAAAAAAA   &   BBBBBBBBBBB   &   CCCCCCCCCCCCCC   \\
AAAAAAAAAAAA   &   BBBBBBBBBBB   &   CCCCCCCCCCCCCC   \\
AAAAAAAAAAAA   &   BBBBBBBBBBB   &   CCCCCCCCCCCCCC   \\
AAAAAAAAAAAA   &   BBBBBBBBBBB   &   CCCCCCCCCCCCCC   \\
AAAAAAAAAAAA   &   BBBBBBBBBBB   &   CCCCCCCCCCCCCC   \\
AAAAAAAAAAAA   &   BBBBBBBBBBB   &   CCCCCCCCCCCCCC   \\
AAAAAAAAAAAA   &   BBBBBBBBBBB   &   CCCCCCCCCCCCCC   \\
AAAAAAAAAAAA   &   BBBBBBBBBBB   &   CCCCCCCCCCCCCC   \\
AAAAAAAAAAAA   &   BBBBBBBBBBB   &   CCCCCCCCCCCCCC   \\
AAAAAAAAAAAA   &   BBBBBBBBBBB   &   CCCCCCCCCCCCCC   \\
AAAAAAAAAAAA   &   BBBBBBBBBBB   &   CCCCCCCCCCCCCC   \\
AAAAAAAAAAAA   &   BBBBBBBBBBB   &   CCCCCCCCCCCCCC   \\
AAAAAAAAAAAA   &   BBBBBBBBBBB   &   CCCCCCCCCCCCCC   \\
AAAAAAAAAAAA   &   BBBBBBBBBBB   &   CCCCCCCCCCCCCC   \\
AAAAAAAAAAAA   &   BBBBBBBBBBB   &   CCCCCCCCCCCCCC   \\
AAAAAAAAAAAA   &   BBBBBBBBBBB   &   CCCCCCCCCCCCCC   \\
AAAAAAAAAAAA   &   BBBBBBBBBBB   &   CCCCCCCCCCCCCC   \\
AAAAAAAAAAAA   &   BBBBBBBBBBB   &   CCCCCCCCCCCCCC   \\
AAAAAAAAAAAA   &   BBBBBBBBBBB   &   CCCCCCCCCCCCCC   \\
AAAAAAAAAAAA   &   BBBBBBBBBBB   &   CCCCCCCCCCCCCC   \\
AAAAAAAAAAAA   &   BBBBBBBBBBB   &   CCCCCCCCCCCCCC   \\
AAAAAAAAAAAA   &   BBBBBBBBBBB   &   CCCCCCCCCCCCCC   \\
AAAAAAAAAAAA   &   BBBBBBBBBBB   &   CCCCCCCCCCCCCC   \\
AAAAAAAAAAAA   &   BBBBBBBBBBB   &   CCCCCCCCCCCCCC   \\
AAAAAAAAAAAA   &   BBBBBBBBBBB   &   CCCCCCCCCCCCCC   \\
AAAAAAAAAAAA   &   BBBBBBBBBBB   &   CCCCCCCCCCCCCC   \\
\end{longtable}


\section{算法环境}
模板中使用 \texttt{algorithm2e} 宏包实现算法环境。关于该宏包的具体用法,
请阅读宏包的官方文档。

\begin{algorithm}[htbp]
\SetAlgoLined
\KwData{this text}
\KwResult{how to write algorithm with \LaTeX2e }

initialization\;
\While{not at end of this document}{
    read current\;
    \eIf{understand}{
        go to next section\;
        current section becomes this one\;
    }{
        go back to the beginning of current section\;
    }
}
\caption{算法示例1}
\label{algo:algorithm1}
\end{algorithm}

\IncMargin{1em}
\begin{algorithm}
\SetKwData{Left}{left}\SetKwData{This}{this}\SetKwData{Up}{up}
\SetKwFunction{Union}{Union}\SetKwFunction{FindCompress}{FindCompress}
\SetKwInOut{Input}{input}\SetKwInOut{Output}{output}

\Input{A bitmap $Im$ of size $w\times l$}
\Output{A partition of the bitmap}
\BlankLine
\emph{special treatment of the first line}\;
\For{$i\leftarrow 2$ \KwTo $l$}{
    \emph{special treatment of the first element of line $i$}\;
    \For{$j\leftarrow 2$ \KwTo $w$}{\label{forins}
        \Left$\leftarrow$ \FindCompress{$Im[i,j-1]$}\;
        \Up$\leftarrow$ \FindCompress{$Im[i-1,]$}\;
        \This$\leftarrow$ \FindCompress{$Im[i,j]$}\;
        \If(\tcp*[h]{O(\Left,\This)==1}){\Left compatible with \This}{\label{lt}
            \lIf{\Left $<$ \This}{\Union{\Left,\This}}
            \lElse{\Union{\This,\Left}}
        }
        \If(\tcp*[f]{O(\Up,\This)==1}){\Up compatible with \This}{\label{ut}
        \lIf{\Up $<$ \This}{\Union{\Up,\This}}
        \tcp{\This is put under \Up to keep tree as flat as possible}\label{cmt}
        \lElse{\Union{\This,\Up}}\tcp*[h]{\This linked to \Up}\label{lelse}
        }
    }
    \lForEach{element $e$ of the line $i$}{\FindCompress{p}}
}
\caption{算法示例2}\label{algo_disjdecomp}
\label{alog:algorithm2}
\end{algorithm}\DecMargin{1em}


\section{代码环境}
模板中使用 \texttt{listings} 宏包实现代码环境。详细用法见宏包的官方说明文档。

以下是代码示例,可以在文中任意位置引用\autoref{first-code} 。
\begin{lstlisting}[language=C, caption=示例代码, label={code:first-code}]
#include <stdio.h>

int main( )
{
    printf("hello, world\n");
    return 0;
}
\end{lstlisting}




\section{引用文献标注}

\subsection{著者-出版年制标注法}

\noindent
\verb|\citestyle{ustcauthoryear}|
\citestyle{ustcauthoryear}

\noindent
\begin{tabular}{l@{\quad$\Rightarrow$\quad}l}
  \verb|\cite{knuth86a}| & \cite{knuth86a}\\
  \verb|\citet{knuth86a}| & \citet{knuth86a}\\
  \verb|\citet[chap.~2]{knuth86a}| & \citet[chap.~2]{knuth86a}\\[0.5ex]
  \verb|\citep{knuth86a}| & \citep{knuth86a}\\
  \verb|\citep[chap.~2]{knuth86a}| & \citep[chap.~2]{knuth86a}\\
  \verb|\citep[see][]{knuth86a}| & \citep[see][]{knuth86a}\\
  \verb|\citep[see][chap.~2]{knuth86a}| & \citep[see][chap.~2]{knuth86a}\\[0.5ex]
  \verb|\citet*{knuth86a}| & \citet*{knuth86a}\\
  \verb|\citep*{knuth86a}| & \citep*{knuth86a}\\
\end{tabular}

\noindent
\begin{tabular}{l@{\quad$\Rightarrow$\quad}l}
  \verb|\citet{knuth86a,tlc2}| & \citet{knuth86a,tlc2}\\
  \verb|\citep{knuth86a,tlc2}| & \citep{knuth86a,tlc2}\\
  \verb|\cite{knuth86a,knuth84}| & \cite{knuth86a,knuth84}\\
  \verb|\citet{knuth86a,knuth84}| & \citet{knuth86a,knuth84}\\
  \verb|\citep{knuth86a,knuth84}| & \citep{knuth86a,knuth84}\\
\end{tabular}

\subsection{顺序编码制标注法}

\noindent
\verb|\citestyle{ustcnumerical}|
\citestyle{ustcnumerical}

\noindent
\begin{tabular}{l@{\quad$\Rightarrow$\quad}l}
  \verb|\cite{knuth86a}| & \cite{knuth86a}\\
  \verb|\citet{knuth86a}| & \citet{knuth86a}\\
  \verb|\citet[chap.~2]{knuth86a}| & \citet[chap.~2]{knuth86a}\\[0.5ex]
  \verb|\citep{knuth86a}| & \citep{knuth86a}\\
  \verb|\citep[chap.~2]{knuth86a}| & \citep[chap.~2]{knuth86a}\\
  \verb|\citep[see][]{knuth86a}| & \citep[see][]{knuth86a}\\
  \verb|\citep[see][chap.~2]{knuth86a}| & \citep[see][chap.~2]{knuth86a}\\[0.5ex]
  \verb|\citet*{knuth86a}| & \citet*{knuth86a}\\
  \verb|\citep*{knuth86a}| & \citep*{knuth86a}\\
\end{tabular}

\noindent
\begin{tabular}{l@{\quad$\Rightarrow$\quad}l}
  \verb|\citet{knuth86a,tlc2}| & \citet{knuth86a,tlc2}\\
  \verb|\citep{knuth86a,tlc2}| & \citep{knuth86a,tlc2}\\
  \verb|\cite{knuth86a,knuth84}| & \cite{knuth86a,knuth84}\\
  \verb|\citet{knuth86a,knuth84}| & \citet{knuth86a,knuth84}\\
  \verb|\citep{knuth86a,knuth84}| & \citep{knuth86a,knuth84}\\
  \verb|\cite{knuth86a,knuth84,tlc2}| & \cite{knuth86a,knuth84,tlc2}\\
\end{tabular}

\subsection{其他形式的标注}

\noindent
\begin{tabular}{l@{\quad$\Rightarrow$\quad}l}
  \verb|\citealt{tlc2}| & \citealt{tlc2}\\
  \verb|\citealt*{tlc2}| & \citealt*{tlc2}\\
  \verb|\citealp{tlc2}| & \citealp{tlc2}\\
  \verb|\citealp*{tlc2}| & \citealp*{tlc2}\\
  \verb|\citealp{tlc2,knuth86a}| & \citealp{tlc2,knuth86a}\\
  \verb|\citealp[pg.~32]{tlc2}| & \citealp[pg.~32]{tlc2}\\
  \verb|\citenum{tlc2}| & \citenum{tlc2}\\
  \verb|\citetext{priv.\ comm.}| & \citetext{priv.\ comm.}\\
\end{tabular}

\noindent
\begin{tabular}{l@{\quad$\Rightarrow$\quad}l}
  \verb|\citeauthor{tlc2}| & \citeauthor{tlc2}\\
  \verb|\citeauthor*{tlc2}| & \citeauthor*{tlc2}\\
  \verb|\citeyear{tlc2}| & \citeyear{tlc2}\\
  \verb|\citeyearpar{tlc2}| & \citeyearpar{tlc2}\\
\end{tabular}
