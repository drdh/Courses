\chapter{软件质量特性}
本节描述了该应用商店系统应当具备的软件质量特性。
\section{功能性}
\subsection{适合性和和准确性}
本系统针对开发者和普通用户的需求进行开发,分为开发者端、客户端和服务器端,应当完全可以满足开发者开发、管理应用和普通用户使用、管理应用的各种需求。
\subsection{保密安全性}
开发者和普通用户均需要登录账户才可以正常地访问数据并进行各种操作(普通用户查询应用和管理本地已安装应用除外),禁止未授权的用户访问和操作。

\section{可靠性}
该应用商店系统能够处理一定的异常,详见3.1功能需求部分的描述。对于未知类型的异常,软件应当能够保存发生异常的状态,使得能够从异常中恢复,并将该未知异常上传给服务器,以便开发人员修复。

\section{易用性}
无论是开发者端、客户端还是服务器端,该系统的各种操作都应当有明确的提示信息和操作说明。另外,对于开发者端和客户端而言,交互界面还应当美观,操作逻辑清晰简洁,便于开发者和普通用户使用。

\section{效率性}
该应用商店系统能够在给定的资源下处理一定的事务,详见3.2性能需求部分的描述。

\section{可维护性}
\subsection{易分析性和易改变性}
该应用商店系统的各个模块应当合理划分,使得当出现问题时能够快速定位问题所在,也使得后续可能的需求变更能够更容易地实现。
\subsection{稳定性}
开发过程应采用增量开发,防止意外修改导致软件失效。
\subsection{易测试性}
该应用商店系统的各个模块的输出应当足够明确和完备,以便确认软件的功能是否正确实现。

\section{可移植性}
\subsection{适应性}
该应用商店系统能够适应不同的平台,包括Windows、Macintosh、Linux、Android、iOS、web等等。
\subsection{易安装性}
该应用商店系统应当提供类似能够“一键安装”的安装包,便于开发者和普通用户使用。


